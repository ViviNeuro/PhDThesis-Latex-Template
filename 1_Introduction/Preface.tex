Every night, we collectively lose consciousness. From an evolutionary perspective, sleep may seem to be the most foolish of biological phenomena, as it disconnects us from the outside world and our own bodies, leaving us vulnerable to predation while our minds wander in the most bizarre places. 

It has long been unclear why we sleep. Interestingly, it’s not just humans who sleep, but most living organisms, including birds, fish, flies, plants, and worms \parencite{siegel_all_2008, zielinski_functions_2016}. Given the ubiquity of sleep across different species, it’s evident that it serves critical functions and offers significant benefits for the organisms. Yet, defining these precise benefits and functions remains a compelling and unresolved question. 

The discovery of the electroencephalogram (EEG) enabled the recording of human brain activity in real time, thus allowing the investigation of the nature of sleep. In 1929, the German psychiatrist Hans Berger demonstrated that the low-voltage activity associated with wakefulness gradually transitions to a higher-voltage and lower-frequency rhythm when the subject falls asleep \parencite{datta_activation_2008}. Soon enough, the sleeping brain was found to be a non-homogeneous state characterised by different sleep stages and oscillatory patterns that serve various functions \parencite{siegel_all_2008, zielinski_functions_2016}. Thanks to the research that has been conducted in the last decades, we now know that sleep is not “\textit{the biggest mistake that evolution has ever made}” \parencite{mignot_why_2008}. Instead, it has been found to be restorative for our brain, body, and mind. Suggested benefits include learning and memory consolidation \parencite{diekelmann_memory_2010,rasch_about_2013}, processing of emotional information and recalibration of emotional brain circuits \parencite{helm_overnight_2010, walker_role_2009}, regulation of metabolism, immune system, and hormones \parencite{mignot_why_2008,zielinski_functions_2016}.
The work presented in this thesis aims to enhance our understanding of the relationship between sleep and the brain, emphasising non-invasive approaches to manipulating sleep to optimise its positive effects on cognition. The focus of this research is on exploring techniques and strategies for manipulating sleep to benefit learning and memory, stimulate creativity, and alleviate the effects of negative emotions. 

Sleep has been extensively demonstrated to be involved in three memory processes: encoding, consolidation, and retrieval. As compared to memory encoding and retrieval, which are best served when the brain is awake, memory consolidation benefits from the decreased level of sensory processing during sleep \parencite{diekelmann_labile_2011,ellenbogen_role_2006,van_der_heijden_sleep_2022}. Notably, during consolidation, memories are not only strengthened but also integrated into pre-existing knowledge networks, transformed, and restructured thanks to a process that involves the repeated reactivation of memory traces during sleep \parencite{diekelmann_memory_2010,rasch_about_2013}. Despite initial theories that offered a passive view of sleep and believed that it consisted of a time when the brain shuts down, it is now widely accepted that sleep is an active time for the brain. In fact, during sleep, the brain selectively reactivates memory by replaying specific patterns of neuronal activity similar to those observed during awake learning \parencite{skaggs_replay_1996, wilson_reactivation_1994}. This replay appears to be critical for the transfer of information from the hippocampus to the neocortex, where it becomes integrated into pre-existing long-term memories \parencite{bergmann_sleep_2012, maquet_experience-dependent_2000, peigneux_are_2004,peigneux_neuroimaging_2015, rasch_about_2013,schonauer_decoding_2017,zhang_electrophysiological_2018}. This process also seems to be tied to a qualitative transformation of memories and indeed, sleep has been shown to facilitate the abstraction of general rules \parencite[e.g.,][]{durrant_sleep-dependent_2011,ellenbogen_human_2007,wagner_sleep_2004}, the integration of distinct elements into unified concepts \parencite[e.g.,][]{lewis_overlapping_2011} and the emergence of false memories \parencite[e.g.,][]{payne_role_2009}. Beyond integration, memory representations can be disintegrated and recombined, allowing associative thinking and creativity \parencite{cai_rem_2009,monaghan_sleep_2015,sio_sleep_2013} and the processing of emotional memory \parencite{helm_overnight_2010,hutchison_targeted_2021,walker_role_2009}. Nowadays, many studies in both animals and humans make use of a technique referred to as targeted memory reactivation (TMR) to manipulate memory processing during sleep. TMR pairs sensory cues with a memory during awake encoding and then presents them again during sleep to bias memory consolidation \parencite{hu_promoting_2020,oudiette_upgrading_2013,rasch_odor_2007}.

The goal of this thesis is to investigate ways of manipulating sleep to enhance cognition. In this general introduction, I will begin by giving a comprehensive summary of the present state of knowledge of sleep physiology and its oscillatory patterns. Next, I will delve into the link between sleep and memory, beginning with an exploration of memory systems and processes, followed by an introduction to the most prominent models of sleep and memory. Emphasis will be placed on the reactivation of memories during sleep. Finally, I will conclude by identifying the central questions that this thesis intends to answer.