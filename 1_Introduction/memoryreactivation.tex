Before diving into different studies conducted on animals and humans on memory reactivation, it is crucial to clarify some technical terms. Genzel and colleagues provided definitions to facilitate the use of a common language among researchers \parencite{genzel_consensus_2020}. As per their definitions, the term reactivation refers to the re-emergence of a previously encoded pattern at a later point in time, whereas replay directly assesses the temporal structure of the sequential information \parencite{genzel_consensus_2020}. This terminology will be used throughout the thesis. 

\subsection{Animal studies}
The first evidence of spontaneous memory replay during sleep is found in the rodent literature on place cells. Place cells are hippocampal CA1 pyramidal neurons that fire selectively when the animal occupies a specific location in the environment, known as place fields \parencite{girardeau_hippocampal_2011,okeefe_hippocampus_1971}. By following the sequence of place cells, the rat’s movements from one location to another can be traced. 
Pavlides and Winson demonstrated that during subsequent sleeping states, hippocampal place cells used in recent awake exploration exhibited an increased firing rate compared to cells that were not involved in the pre-sleep exploration \parencite{pavlides_influences_1989}. In the seminal study by Wilson and McNaughton, the activity of pairs of hippocampal place cells was recorded during both a spatial navigation task in a maze and during the SWS period that preceded and followed the task \parencite{wilson_reactivation_1994}. Notably, during the SWS period that followed the task, the same place cells that fired during exploratory behaviour fired together, indicating that information acquired during active behaviour is re-expressed in hippocampal circuits during sleep. The order of neuronal firing observed during wakefulness was preserved during sleep reactivation, albeit in a compressed temporal manner \parencite{wilson_reactivation_1994}. \\
Reactivation during sleep has also been observed in various brain regions and species, thereby supporting the Active System Consolidation Hypothesis (see section \ref{Intro:sec:Models of sleep and memory}). For instance, visual \parencite{ji_coordinated_2007}, parietal \parencite{qin_memory_1997}, prefrontal \parencite{benchenane_coherent_2010,euston_fast-forward_2007,johnson_stored-trace_2010,peyrache_replay_2009} and sensorimotor cortices \parencite{hoffman_coordinated_2002} have all shown reactivation. 
Evidence of reactivation and replay during REM sleep is substantially less extensive compared to NREM, however the occurrence of these phenomena in rodents is supported by place cell recordings \parencite{louie_temporally_2001,poe_experience-dependent_2000}, appetitive conditioning \parencite{maho_appetitive_2002} and cell recording from the primary motor cortex \parencite{eckert_neural_2020}.
Interestingly, while replay during NREM sleep occurs in a temporally compressed form \parencite{lee_memory_2002}, REM sleep replay is temporally structured at a timescale closer to that seen during wakefulness \parencite{louie_temporally_2001}. Just like replay during sleep, wake replay is triggered by SW-Rs however, the order during awake replay is reversed compared to that seen during sleep  \parencite{foster_reverse_2006}. 





\subsection{Human studies}
Studies investigating memory reactivation during sleep in humans are limited due to the low temporal and spatial resolution of non-invasive techniques compared to electrophysiological recordings \parencite{schreiner_electrophysiological_2020}. 
Early attempts to reveal memory reactivation during sleep in humans, investigated whether brain regions activated during memory encoding show corresponding activity during subsequent sleep \parencite{bergmann_sleep_2012,maquet_experience-dependent_2000,peigneux_are_2004,yotsumoto_location-specific_2009}. For instance, using simultaneous EEG-fMRI recordings Bergmann and colleagues observed that participants who learned face-scene associations exhibited increased activity in both hippocampal and neocortical areas compared to those who performed a visuomotor control task. The increased neural activation was temporally coupled with increased spindle amplitude and correlated with pre-sleep behavioural performance, indicating the involvement of spindles in reactivation-like patterns \parencite{bergmann_sleep_2012}. Another study employed positron emission topography (PET) to investigate the effects of training on a probabilistic serial reaction time task. Findings indicated that brain areas activated during the task in wakefulness were similarly active during REM after the training, thus supporting the hypothesis that there is an experience-dependent re-activation of specific brain areas during post-training REM sleep \parencite{maquet_experience-dependent_2000}.
The advent of multivariate pattern analysis \parencite[MVPA;][]{haxby_distributed_2001} and representational similarity analysis \parencite[RSA;][]{schonauer_decoding_2017} helped tackle the issue of whether the re-expression of encoding related activity reflects the content of the learned task. In other words, studies that used these techniques aimed to measure replay of stimulus-related activity patterns \parencite{deuker_memory_2013,liu_human_2019,schonauer_decoding_2017,sterpenich_memory_2014,zhang_electrophysiological_2018}. For instance, Schönauer and colleagues used MVPA to decode EEG activity during sleep to determine whether participants viewed faces or houses during a declarative task performed the day before. They showed that the memory content was reprocessed during both NREM and REM sleep and that reactivation of declarative material during NREM sleep was associated with a greater memory performance \parencite{schonauer_decoding_2017}. Furthermore, combining intracranial EEG electrodes with RSA in epilepsy patients, Zhang and colleagues identified that stimulus-specific activity spontaneously re-occurred during waking rest and sleep, but only ripple-triggered replay during NREM sleep was associated with memory consolidation \parencite{zhang_electrophysiological_2018}.\\
These studies support the evidence that the spontaneous reactivation of prior learned material during sleep constitutes a plausible mechanism supporting sleep-based memory consolidation. However, only methods that can directly manipulate memory reactivation during sleep provide evidence for a causal role of sleep in memory consolidation. Targeted Memory Reactivation (TMR) is a technique that emerged to directly address this issue.




\subsection{Targeted Memory Reactivation (TMR)}
TMR involves associating learning materials used in a task during awake learning with a contextual cue, like a sound or an odour. These stimuli are then unobtrusively re-presented during sleep, usually during specific sleep stages, to bias the spontaneous memory reactivation process towards the cued stimuli.  Performance change scores between reactivated and non-reactivated items are then compared during a post-sleep test. The aim of this technique is to selectively improve memory consolidation of the cued items and thereby increase performance on those items in subsequent memory tests \parencite{andrillon_sleep_2011,hu_promoting_2020,rasch_odor_2007}.
Early attempts with the use of TMR date back to the end of the 1980s \parencite{oudiette_upgrading_2013}, however, this procedure was experimentally demonstrated for the first time by Rasch and colleagues in 2007 \parencite{rasch_odor_2007}. In this seminal study, olfactory stimuli were used to cue declarative memory during sleep. Participants were trained on a two-dimensional memory task involving object locations while smelling the scent of a rose. During the subsequent night, the same odour was presented again during SWS periods without disrupting participants’ sleep. Declarative memory improvement was measured the next day by comparing the recall performance of participants who did or did not receive the rose-scented air during sleep. Those who received the TMR procedure demonstrated enhanced recall performance for the cued stimuli compared to a control group. Additionally, functional magnetic resonance imaging (fMRI) revealed a higher hippocampal activation for those object-location pairs that were re-exposed to the odour during SWS \parencite{oudiette_upgrading_2013,rasch_odor_2007}. However, it remained unclear from this study whether TMR during sleep has the ability to selectively reactivate and strengthen specific memories formed during a learning episode. \\
Rudoy and colleagues \parencite{rudoy_strengthening_2009} addressed the specificity issues by using auditory cues instead of sounds. Participants learned semantic associations between images and sounds (e.g., cat/meow) in specific locations. During the TMR procedure delivered during SWS, half of the items were cued by representing the sound of the corresponding objects. Post-sleep results showed higher accuracy for the items that were cued during sleep, suggesting that specific memories can be targeted and strengthened during sleep \parencite{rudoy_strengthening_2009}. \\
These findings have been further supported by animal studies. Bendor and Wilson trained four rats on an auditory-spatial association task in which sounds (sound L or R) were associated with reward locations: rats received a food pellet reward at the left-end side of a track for sound L and at the right-end side for sound R. In addition, control sounds not associated with the behavioural task were also played. During subsequent sleep, auditory cues were represented during NREM sleep, and place cell activity in the CA1 region of the hippocampus was recorded during the task and during sleep. After comparing firing rates for each acoustic stimulus during NREM periods, they observed that both task-related cues (sound L and R) biased the content of replay events. Sound L resulted in a higher firing rate of individual place cells and ensembles of place cells with place fields on the left side, while sound R had the same effect on the right side \parencite{bendor_biasing_2012}.

TMR has been applied during both NREM (N2 and SWS) and REM sleep however, to date, a relatively low number of studies examined the effect of TMR during REM sleep. The majority of studies applied TMR during SWS to boost performance on declarative memory tasks. Beyond spatial location \parencite{diekelmann_labile_2011,diekelmann_offline_2012,rasch_odor_2007,rudoy_strengthening_2009}, TMR studies have shown a benefit for spatial navigation \parencite{shimizu_closed-loop_2018}, vocabulary learning \parencite{schreiner_boosting_2015}, associative learning tasks \parencite{cairney_mechanisms_2017,cairney_memory_2018}, word recall \parencite{fuentemilla_hippocampus-dependent_2013}. Additionally, evidence shows that SWS TMR can improve non-declarative memories such as procedural skills \parencite{antony_cued_2013,cousins_cued_2014,cousins_cued_2016,schonauer_strengthening_2013} and emotional memories \parencite{cairney_targeted_2014,lehmann_emotional_2016,hu_promoting_2020}.\\
TMR during N2 has been mainly employed with motor skill learning tasks \parencite{laventure_nrem2_2016,laventure_beyond_2018} due to the well-known role played by sleep spindles in the consolidation of motor memories traces \parencite{fogel_function_2011,morin_motor_2008}. 
Not many TMR studies focused on REM sleep. Early studies conducted with animals, by Hennevin and Hars, reported significant performance improvement in rats on an active avoidance conditioning task after receiving ear shocks as a cue during post-learning REM sleep \parencite{hars_improvement_1985,hennevin_is_1987}. Early attempts in humans also demonstrated enhanced memory after auditory stimulation during REM on a Morse code task \parencite{guerrien_enhancement_1989} and a complex logic task \parencite{smith_post_1990}. On the contrary, more recent attempts have reported negative REM sleep-declarative and procedural memory effects \parencite{laventure_nrem2_2016,rasch_odor_2007} and a positive REM-emotional memory effect \parencite{hutchison_targeted_2021,schwartz_enhancing_2022,wassing_restless_2019}.\\
Today’s research takes advantage of artificial intelligence methods to develop automated sleep stage scoring and to perform TMR in participants’ own homes, thus allowing real-life applications. Recently, Whitmore and colleagues developed a TMR system called “SleepStim” that works with a smartphone and a smartwatch used to play sounds and record movement and heart rate parameters, respectively \parencite{whitmore_improving_2022}. They tested whether at-home TMR, delivered via the SleepStim system during N3, replicates the spatial-memory benefit observed in laboratory settings. Across two experiments, they found a stronger TMR effect using a relatively low auditory cue intensity \parencite{whitmore_improving_2022}. Furthermore, TMR manipulations can be extended to clinical settings and used in clinical psychotherapy. Some studies explored the potential of TMR to weaken fear memories \parencite{hauner_stimulus-specific_2013,oudiette_fear_2014} and to reduce negative valence of stimuli \parencite{hutchison_targeted_2021,rihm_replay_2015}, which could be used to treat emotion and memory-related disorders, such as depression and PTSD. For example, a clinical reduction in nightmare frequency and more positive dream emotions have been demonstrated in adults with nightmare disorder (ND) after exposing them to TMR during REM sleep for 14 days in combination with imagery rehearsal therapy \parencite{schwartz_enhancing_2022}. 

\subsection{Research objectives}
This thesis comprises three experimental chapters that investigate the beneficial effects of sleep manipulation on cognition. 

\textbf{Chapter 2 }aimed to examine whether blocking natural light during nighttime sleep with the use of an eye mask benefits memory and alertness. To this end, we compared participants’ performance on a cognitive battery while wearing an eye mask or a control mask during sleep. Additionally, we utilised a wearable EEG device to track sleep architecture, as we were interested in looking at the impact of the eye mask manipulation on sleep parameters.

\textbf{Chapter 3} explored the role of sleep in consolidating the effects of a cognitive training, thereby overcoming functional fixedness, a cognitive obstacle that prevents people from thinking outside the box and reaching insight. Participants were first exposed to a training session and then their performance was compared after a period of nocturnal sleep or daytime wakefulness. 

In \textbf{Chapter 4}, the focus switched to assessing the potential of TMR during REM sleep to disarm the effects of negative emotions. We examined the effect of cueing on subjective and objective measurement of arousal. To this end, we employed subjective ratings of arousal, functional MRI, and heart rate deceleration.  


