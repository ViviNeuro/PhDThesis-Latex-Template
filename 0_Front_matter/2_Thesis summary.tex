\markboth{}{}
\chapter{Thesis summary}
On average, we spend one-third of our lives asleep, and we have little idea why. 
Despite the importance of sleep to overall health, sleep has been neglected for decades and considered an inactive state in which the brain “turns off” to rest from daily activities. However, there is now compelling evidence that sleep plays a pivotal role in various domains, including learning and memory, physical and mental wellbeing. The work described in this thesis is centred around exploring non-invasive ways of manipulating sleep to enhance cognition. 

\textbf{Chapter 2} delves into the effects of wearing an eye mask to block out light during sleep and its implications for daily life. This simple and cost-effective manipulation resulted in enhanced reaction times and better memory encoding compared to a control condition. Such improvements are particularly advantageous in situations demanding rapid reflexes, like driving. Furthermore, the benefits can extend to academic and professional spheres, leading to enhanced performance across diverse tasks.

\textbf{Chapter 3} investigates whether sleep facilitates insight problem solving. We found that offline consolidation and reorganisation of memories had a beneficial effect on insight, but this result was confounded by the influence of circadian rhythms.

Finally,\textbf{ Chapter 4} explores the potential benefits of an experimental technique called targeted memory reactivation (TMR) applied during rapid eye movement (REM) sleep for arousal processing. Our manipulation resulted in a reduction of emotional reactivity, as demonstrated by objective measurements of arousal.  Notably, the effect of cueing on subjective arousal responses was tied to participants’ baseline arousal levels. 

In conclusion, this thesis provides valuable insights into the importance of sleep in enhancing cognitive functions and sheds light on non-invasive interventions whose implications extend far beyond the laboratory and into everyday life.
